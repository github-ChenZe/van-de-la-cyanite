% Mathematics Include


\hypersetup{%
  colorlinks=false,% hyperlinks will be black
  linkbordercolor=blue,% hyperlink borders will be red
  pdfborderstyle={/S/U/W 1}% border style will be underline of width 1pt
}

\usepackage{wrapfig}
\usepackage{leftidx}
\usepackage{dashrule}
\usepackage{ifthen}
\usepackage{blindtext}


\usepackage{xkeyval}
\usepackage{forloop}

% Physics Include



\usepackage{mdframed}
\usepackage{lipsum}% just to generate text for the example



\usepackage{longtable}


\newcommand{\positiontextbox}[4][]{%
  \begin{tikzpicture}[remember picture,overlay]
%    \draw[step=0.5,gray!80!white] (current page.north west) grid (current page.south east); % For controlling
    \node[inner sep=10pt,right,line width=0pt,#1] at ($(current page.north west) + (#2,-#3)$) {#4};
  \end{tikzpicture}%
}

\newmdenv[
  topline=false,
  bottomline=false,
  skipabove=\topsep,
  skipbelow=\topsep
]{siderules}


\newcommand{\bu}[3]{#1_{#2}^{\pare{#3}}}
\newcommand{\axref}[1]{公理\ref{ax:#1}}
\newcommand{\dref}[1]{定义\ref{def:#1}}
\newcommand{\Dref}[1]{定义\ref{def:#1}{(#1)}}
\newcommand{\aref}[1]{假设\ref{ass:#1}}
\newcommand{\tref}[1]{定理\ref{thm:#1}}
\newcommand{\Tref}[1]{定理\ref{thm:#1}(#1)}
\newcommand{\lref}[1]{引理\ref{lem:#1}}
\newcommand{\LMref}[1]{引理\ref{lem:#1}(#1)}
\newcommand{\cref}[1]{推论\ref{coll:#1}}
\newcommand{\Cref}[1]{推论\ref{coll:#1}(#1)}
\newcommand{\pref}[1]{命题\ref{prp:#1}}
\newcommand{\Pref}[1]{命题\ref{prp:#1}{(#1)}}
\newcommand{\rmref}[1]{附注\ref{rm:#1}}
\newcommand{\eref}[1]{例\ref{ex:#1}}
\newcommand{\Eref}[1]{例\ref{ex:#1}{#1}}
\newcommand{\reflref}[1]{反射\ref{reflex:#1}}
\newcommand{\dcompare}[1]{\textit{平行于\dref{#1}}}
\newcommand{\tcompare}[1]{\textit{平行于\tref{#1}}}
\newcommand{\lcompare}[1]{\textit{平行于\lref{#1}}}
\newcommand{\ecompare}[1]{\textit{平行于\eref{#1}}}
\newcommand{\ccompare}[1]{\textit{平行于\cref{#1}}}
\newcommand{\inter}[1]{\mathring{#1}}
\newcommand{\forest}[3]{对于{#1},存在{#2},使得{#3}}
\newcommand{\tuno}{$T_1$公理}


\newcommand{\hd}{H\"{o}lder}

\renewcommand{\proofname}{证明}
\newcommand{\prop}{\propto}


\newcommand{\refl}[1]{\vspace{0.5em}\par\noindent\fbox{%
    \parbox{0.97\textwidth}{%
    \begin{reflection}
        #1
    \end{reflection}
    }%
}\vspace{0.5em}\par}
\newcommand{\rref}[1]{反射\ref{refl:#1}}
\newcommand{\tbref}[1]{表\ref{table:#1}}
\allowdisplaybreaks

\newenvironment{aenum}{\begin{enumerate}[label=\textnormal{(\alph*)}]}{\end{enumerate}}
\newcommand{\sus}[1]{\vspace{1em}\hrule\vspace{1em}{\color{red}#1}\vspace{1em}\hrule\vspace{1em}}
\newcommand{\sep}{\par\noindent\hdashrule{\textwidth}{1pt}{1pt}}
\newcommand{\offlegend}[1]{\tilde{P}_{#1}}

\newcommand{\ceiling}[1]{\lceil #1 \rceil}
\newcommand{\floor}[1]{\lfloor #1 \rfloor}



% Physics Head

\newcommand{\nbh}[1]{U\pare{#1}}
\newcommand{\pnbh}[1]{\accentset{\circ}{U}\pare{#1}}
\newcommand{\snbh}[2]{U_{#1}\pare{#2}}
\newcommand{\psnbh}[2]{\accentset{\circ}{U}_{#1}\pare{#2}}
%\newcommand{\mapsto}{\leadsto}
\newcommand{\mie}{\mathrm{i.e.}}

\newcommand{\proofpara}[1]{{\textit{#1}}}

\usepackage{environ}

\NewEnviron{erefl}{
\begin{wrapfigure}{r}{.5\textwidth}
\begin{tcolorbox}
[width=.5\textwidth,standard jigsaw,colframe=red,
opacityframe=0.5, opacitybacktitle=0.5,
title filled, title=反射]
\BODY
\end{tcolorbox}
\end{wrapfigure}
}

\newcommand{\fref}[1]{图\ref{fig:#1}}

\newcommand{\patterntitle}[2]{\textbf{#1}(参见#2)}
\newcommand{\patterncf}[1]{(参见#1)}


\makeatletter

\newtcbtheorem[number within=section]{patterntheorem}{反射}%
{standard jigsaw,colframe=blue,opacityframe=0.5, opacitybacktitle=0.5, fonttitle=\bfseries}{pth}

\newcounter{patterncount}
\setcounter{patterncount}{0}

\def\wrapbeginr{\wrapfigure{r}{.5\textwidth}}
\def\wrapendr{\endwrapfigure}
\def\wrapbeginno{}
\def\wrapendno{}
\def\wrapbegin{\wrapbeginno}
\def\wrapend{\wrapendno}
\def\patternlabelnameno{PATTERNLABEL\arabic{patterncount}}
\def\patternlabel{\patternlabelnameno}

\define@cmdkey{patternbox}[patternbox@]{starcount}[none]{}
\define@cmdkey{patternbox}[patternbox@]{starcolor}[red]{}
\define@cmdkey{patternbox}[patternbox@]{wrapbox}[no]{
\def\wrapbegin{\csname wrapbegin#1\endcsname}
\def\wrapend{\csname wrapend#1\endcsname}
}
\define@cmdkey{patternbox}[patternbox@]{label}[]{
\def\patternlabel{#1}
}
\setkeys{patternbox}{starcount=0,starcolor=red,wrapbox=no,label=\patternlabelnameno}

\newenvironment{spattern}[2][]
{%
\setkeys{patternbox}{#1}
\stepcounter{patterncount}
\wrapbegin
\patterntheorem{#2}{\patternbox@label}
%\tcolorbox
%[standard jigsaw,colframe=blue,
%opacityframe=0.5, opacitybacktitle=0.5,
%title filled, title=反射 \arabic{patterncount} \drawnpatternstars{\patternbox@starcount}]
%\patterntitle%
\patterncf
}
{%\endtcolorbox
\endpatterntheorem
\wrapend}

\newenvironment{cpattern}
{
\stepcounter{patterncount}
\tcolorbox
[width=\textwidth,standard jigsaw,colframe=blue,
opacityframe=0.5, opacitybacktitle=0.5,
title filled, title=反射 \arabic{patterncount}]\patterntitle}
{\endtcolorbox}

\newenvironment{pattern}[2]
{
\stepcounter{patterncount}
\wrapfigure{#1}{#2}
\tcbset{enhanced}
\tcolorbox
[width=#2,standard jigsaw,colframe=blue,
opacityframe=0.5, opacitybacktitle=0.5,
title filled, title=反射 \arabic{patterncount}]\patterntitle}
{\endtcolorbox
\endwrapfigure}

\newcolumntype{W}{>{$\displaystyle }c<{$}}
\newcolumntype{V}{>{$}c<{$}}

\newcounter{patternstarcount}
\newcommand{\drawpatternstar}{%
\raisebox{0.25em}{%
\psset{unit=1em}
\pnode(0.25;90){N1}
\pnode(0.25;162){N2}
\pnode(0.25;234){N3}
\pnode(0.25;306){N4}
\pnode(0.25;18){N5}
\pspolygon*[linecolor=\patternbox@starcolor](N3)(N1)(N4)(N2)(N5)%
}%
}
\newcommand{\drawnpatternstars}[1]{
\forloop{patternstarcount}{0}{\value{patternstarcount}<#1}{
\drawpatternstar
}
}

\newcommand{\pswall}{\psframe[linestyle=none,fillstyle=hlines]}

\makeatother

\newcommand{\refsys}[1]{舒'#1}
\newcommand{\refyhw}[1]{杨'#1}

%\newcommand{\erf}{\mathrm{erf}}
\newcommand{\erfi}{\mathrm{erfi}}
%\newcommand{\erfc}{\mathrm{erfc}}
\newcommand{\ehxs}[1]{e^{-\frac{#1^2}{2}}}
\newcommand{\dcol}[2]{\[ \left.#1 \hspace{1em}\right\vert\hspace{1em} #2 \]}
\newcommand{\titlegamma}{\texorpdfstring{$\Gamma$}{Gamma}}
\newcommand{\titleB}{\texorpdfstring{$B$}{B}}
\newcommand{\switch}[2]{\brac{#1 | #2}}
\newcommand{\SYSexeref}[1]{(舒幼生#1)}
\newcommand{\GCexeref}[1]{(普化#1)}
\newcommand{\Lref}[1]{(课堂例题#1)}
\newcommand{\ptref}[1]{反射\ref{pth:#1}}
\newcommand{\warning}[1]{\par\textit{注意:#1}}

% Computer Science Head

\lstset{language=Java}
\newcommand{\snp}[1]{\lstinline!#1!}
\newcommand{\term}[2]{\textbf{#1\ifthenelse{\equal{#2}{}}{}{(#2)}}}

% Chemistry Head
\newcommand{\hvap}{\Delta H_{\mathrm{vap}}}
\newcommand{\cfsee}[1]{(c.f. #1)}


%

\newcommand{\subst}[2]{\left. #1 \right\vert_{#2}}

\def\MARK{\positiontextbox{0.5em}{1em}{\textit{Tour de Force et Fancy}}}

\newcommand\abeq[1]{\stackrel{\mathclap{\normalfont\mbox{#1}}}{=}}

\usepackage{anyfontsize}
